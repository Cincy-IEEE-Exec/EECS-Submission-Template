\documentclass[./subfile.tex]{subfiles}

\title{EECS Submission Template}
% The \IEEEmembership command is being used for a different purpose here.
% Please replace it with Degree, Magor
% The '`' characters are telling LaTeX to insert a space, but don't ever insert a new line there.
% This will make it look better, we swear
\author{
    Tyler~Westland,~\IEEEmembership{PhD,~Comp.~Sci.,}
    Other~Author,~\IEEEmembership{BS,~Elec.~Eng.}
}

\begin{document}
    % DO NOT EDIT/REMOVE
    \maketitle

    % As a general rule, do not put math, special symbols or citations
    % in the abstract or keywords.
    \begin{abstract}
        \LaTeX{} is an academic standard for journal submissions.
        This template article will not only shows what an article might look like, but will answer some questions about how to use \LaTeX.
    \end{abstract}

    \section{Introduction}
        % The very first letter is a 2 line initial drop letter followed
        % by the rest of the first word in caps.
        % 
        % form to use if the first word consists of a single letter:
        % \IEEEPARstart{A}{demo} file is ....
        % 
        % form to use if you need the single drop letter followed by
        % normal text (unknown if ever used by the IEEE):
        % \IEEEPARstart{A}{}demo file is ....
        % 
        % Some journals put the first two words in caps:
        % \IEEEPARstart{T}{his demo} file is ....
        % 
        % Here we have the typical use of a "T" for an initial drop letter
        % and "HIS" in caps to complete the first word.
        \IEEEPARstart{T}{his} is the template for the EECS Student Magazine edited by UC IEEE.
        To get started with this template:
        \begin{enumerate}
            \item Register for an account on Overleaf:  \url{https://www.overleaf.com/register}.
            \item Open the menu, in the top left corner.
            \item Select "Copy Project", under "Actions".
            \item Read this document to learn what is filler/guide and what can be replaced with your awesome project.
        \end{enumerate}
        
        To submit your finished product we would appreciate a zip file of your project sent to our editor.
        \begin{enumerate}
            \item Open the menu, in the top left corner.
            \item Click "Source", under "Downloads".
            \item Email the zip file to our edito at execieee@gmail.com.
        \end{enumerate}
        
    \section{Some \LaTeX{} tips}
    \subsection{Where do I get answers for my \LaTeX{} questions?}
        You have a few options:
        \begin{itemize}
            \item Read \citetitle{lamport1994latex}\cite{lamport1994latex} from \citeyear{lamport1994latex}
                \begin{itemize}
                    \item Probably outdated though.
                \end{itemize}
            \item Read this document
                \begin{itemize}
                    \item Contains some pointers and examples, but not answers to edge questions.
                \end{itemize}
            \item Visit \url{https://www.overleaf.com/learn/latex/Main_Page} and \url{https://www.overleaf.com/learn/latex/Main_Page}.
                \begin{itemize}
                    \item Contains documentation, and lots of answered questions.
                \end{itemize}
            \item Email the editor at execieee@gmail.com.
                \begin{itemize}
                    \item The editor probably just fakes expertise by using Stack Exchange, but is a human who will kindly answer your emails for help.
                \end{itemize}
        \end{itemize}
        
    \subsection{Inserting Figures}
        Inserting figures in \LaTeX{} can be a hassle as it is unlike any other method you have seen before.
        Please see the below example.
        \begin{figure}[!h]
            \centering
            \includegraphics[height=1.0in]{./img/IEEE_logo}
            \caption{The IEEE logo}
            \label{fig_ieeeLogo}
        \end{figure}
        
        Still have questions?
        \url{https://www.overleaf.com/learn/latex/Inserting_Images}
    
    \subsection{Making Citations}
        Citations, references, whatever you call it \LaTeX{} does it better.
        This is done by using the \textit{\\cite\{\(<\)articleLabel\(>\)\}} command.
        Note that \(<\)articleLabel\(>\) is a short name given to whatever you are referencing.
        When your \LaTeX{} file is compiled it will be replaced with the proper reference number.
        Simple, no more mismatched numbers.
        
        Another frustration in citations is the formatting, because it is tedious and a computer should do it.
        \LaTeX{} agrees, so it does it for you!
        You simply need to put the properly formatted information in "bibliography.bib".
        BibTex (a package in \LaTeX{} will take care of the rest).
        Making the properly formatted information is tedious as well, so get a computer to do it for you. The most common way to do this is to search for the article on \url{https://scholar.google.com}. 
        It will offer a properly formatted citation in many formats, as well as Bibtex.
        
        Still have questions?
        See the article on Overleaf \url{https://www.overleaf.com/learn/latex/Articles/Getting_started_with_BibLaTeX}.
        
    \section{Conclusion}
        We hope this template offers guidance on how to use \LaTeX.
        If you need more info on what the content of your submission should look like please read the "Submission Guidelines" at \url{https://www.ieee.uc.edu/eecs-student-journal/} and/or email the editor at execieee@gmail.com.

    % DO NOT EDIT/REMOVE
    \printbibliography
\end{document}
